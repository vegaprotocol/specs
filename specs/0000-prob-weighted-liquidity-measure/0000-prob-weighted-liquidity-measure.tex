\documentclass[10pt]{amsart}

\usepackage[utf8]{inputenc}
\usepackage[a4paper, lmargin=3cm, rmargin=3cm, tmargin=3cm, bmargin=2.8cm]{geometry}
% \usepackage{charter}
\usepackage{mathpazo}
% \usepackage{newcent}
\usepackage{amsmath}
\usepackage{amssymb}
\usepackage{amsthm}
\usepackage{bm} % for bold greek letters
\usepackage{ifthen}
\usepackage{paralist}
\usepackage{dsfont}
\usepackage{graphicx}
\usepackage[UKenglish]{babel}
\usepackage{epigraph}
\usepackage{bbm}
\usepackage{todonotes}
\usepackage[title, titletoc]{appendix}
\usepackage{titlesec}
\usepackage{etoolbox}
\usepackage{tocloft}
\usepackage[normalem]{ulem}
\usepackage{fancyhdr}
\usepackage{caption}
\usepackage{textcomp}
\usepackage{float}
% \usepackage{parskip}  % uncomment to remove paragraph indents
%\usepackage{refcheck}

% We don't want to see subsections of the appendices in the ToC
\appto\appendices{\addtocontents{toc}{\protect\setcounter{tocdepth}{1}}}
\appto\listoffigures{\addtocontents{lof}{\protect\setcounter{tocdepth}{1}}}
\appto\listoftables{\addtocontents{lot}{\protect\setcounter{tocdepth}{1}}}
\addtocontents{toc}\protect\setcounter{tocdepth}{2}

% Make bullets a little smaller
\renewcommand\labelitemi{$\vcenter{\hbox{\footnotesize$\bullet$}}$}

% Set up hyperlinks. Blue seems to be the only non-sucky default colour.
%\definecolor{DarkBlueLinks}{RGB}{10,85,145}
\definecolor{DarkBlueLinks}{RGB}{5,32,144} % more blue, less green

\usepackage{hyperref}
\hypersetup{
    colorlinks=true,
    linkcolor=DarkBlueLinks,
    filecolor=blue,      
    urlcolor=blue,
    citecolor=DarkBlueLinks,
    pdftitle={Radical Liquidity on Vega},
  	pdfauthor={W. G. and B. M. and  T. R. and D. \v{S}},
}

\title{WIP: Liquidity Measure}

\date{
%    \vspace{2em}
    \today\\
%    \vspace{0.5em}
}




\DeclareMathOperator*{\argmin}{arg\,min}
\DeclareMathOperator*{\argmax}{arg\,max}




\begin{document}
\maketitle
\section*{Acceptance Criteria}

To be provided later.

\section*{Summary}
We need to measure liquidity available on a market in order to see whether market makers are keeping 
their commitment. 
Here we propose a method counts liquidity as the probability weighted average of volume on the book. 
This gives view of liquidity at one instant of time; we then use exponential weighted average over time to obtain the desired measure.

\section*{Terminology}
\begin{itemize}
\item $\Lambda_t$ is the exponentially-in-time weighted and probabilistically-in-space-weighted liquidity which we are defining in this spec file.
\item  mid price = (best bid - best offer) / 2 (or undefined if either side of the book is empty)
\item buy / sell side volume refer to the volume available at a distance from mid price, $V = V(x)$, where $x > 0$ refers to sell side, $x < 0$ refers to buy side and $x$ belongs to the set of all price points available on the book
\item probability of volume at distance from mid price being hit: $p = p(x)$, this will come from risk model
\item auction level buy price $x_{min} < 0$ and auction level sell price $x_{max} > 0$ will come from risk model together with market parameter specifiyng what percentile move triggers auction  
\item instantenaous liquidity $\lambda_t$, defined below in detail.
\item decay parameter $\delta$ which determines how far back in time do we go when averaging instantenaous liquidity.
\item weighting parameter $\alpha$ which determines how steep the exponential decay is.

\end{itemize}


Note that both $\delta$ and $\alpha$  are network wide parameters.
	

\section*{Details}
Auction periods should be ignored during calculations (so we pretend time stops).

The instantenaous liquidiy should get calculated periodically.
To be more precise this should be after each "event" if a configurable liquidity time step parameter has been exceeded. 
We will need to keep track of the instantenaous liquidity $\lambda_{t_k}$ and the period of time $[t_k,t_{k+1})$ for which it was calculated.

\subsection*{Calculating the instantenaous liquidity}

Assume that time now is $t = t_k$. We wish to calculate $\lambda_t$.

Case 1: no mid price
$\lambda_t := 0$ if there is no mid price (i.e. when either the buy or sell side of the book are empty)

Case 2: we have mid price
1. Obtain $x_{min}$ and $x_{max}$ from the risk model. 
1. Get the list of possible $x$ s.t. $x_{max} \geq x > 0$ values from the order book. Call these $x^+_i$, with $i = 1,\ldots,N^+$. 
1. Get the list of possible $x$ s.t. $x_{min} \leq x < 0$ from the order book and call these $x^-_i$, with $i = 1, \ldots , N^-$. 
1. Get the volume $V(x)$ available at each $x = x^-_i$ and $x^+_i$ from the order book.
1. Get the probability $p(x)$ for each of $x = x^-_i$ and $x^+_i$ from the risk model. 

Note that in all the above the $x$ is relative to the mid-price and so you may need to perform the requisite transformations.

Now you can calculate 
$$
\lambda_t := 
\min\left(
    \sum_{i=1}^{N^+} V(x^+_i) p(x^+_i), 
    \sum_{i=1}^{N^-} V(x^-_i) p(x^-_i), 
\right)\,.
$$

\subsection*{Calculating the time average}

We now have a list of $\lambda_{t_k}$ and the corresponding time periods $[t_k,t_{k+1})$ from the market inception until now (time now is $t$). 
From this we get $\lambda_t$ for any $t$ using peicewise constant extrapolation i.e. $\lambda_s = \lambda_{t_k}$ whenever $s \in [t_k, t_{k+1})$.

We then have 
$$
\Lambda_t := \int_{t-\delta}^t e^{\alpha (s - (t-\delta))} \lambda_s \,ds\,.
$$
Note that the integral above is just a sum but it's an easy way to express over which indices $k$ we should sum up.


\section*{Pseudo-code / Examples}

To be provided later.

\section*{Test cases}

To be provided later.

\end{document}